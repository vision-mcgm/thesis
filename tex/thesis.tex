\documentclass[a4paper]{article}
\title{PhD Thesis}
\author{Fintan S. Nagle}



%BEGIN MATRIX GUMF
% Load TikZ
\usepackage{tikz}
\usetikzlibrary{matrix,decorations.pathreplacing,calc}

% Set various styles for the matrices and braces. It might pay off to fiddle around with the values a little bit
\pgfkeys{tikz/mymatrixenv/.style={decoration=brace,every left delimiter/.style={xshift=3pt},every right delimiter/.style={xshift=-3pt}}}
\pgfkeys{tikz/mymatrix/.style={matrix of math nodes,left delimiter=[,right delimiter={]},inner sep=2pt,column sep=1em,row sep=0.5em,nodes={inner sep=0pt}}}
\pgfkeys{tikz/mymatrixbrace/.style={decorate,thick}}
\newcommand\mymatrixbraceoffseth{0.5em}
\newcommand\mymatrixbraceoffsetv{0.2em}

% Now the commands to produce the braces. (I'll explain below how to use them.)
\newcommand*\mymatrixbraceright[4][m]{
    \draw[mymatrixbrace] ($(#1.north west)!(#1-#3-1.south west)!(#1.south west)-(\mymatrixbraceoffseth,0)$)
        -- node[left=2pt] {#4} 
        ($(#1.north west)!(#1-#2-1.north west)!(#1.south west)-(\mymatrixbraceoffseth,0)$);
}
\newcommand*\mymatrixbraceleft[4][m]{
    \draw[mymatrixbrace] ($(#1.north east)!(#1-#2-1.north east)!(#1.south east)+(\mymatrixbraceoffseth,0)$)
        -- node[right=2pt] {#4} 
        ($(#1.north east)!(#1-#3-1.south east)!(#1.south east)+(\mymatrixbraceoffseth,0)$);
}
\newcommand*\mymatrixbracetop[4][m]{
    \draw[mymatrixbrace] ($(#1.north west)!(#1-1-#2.north west)!(#1.north east)+(0,\mymatrixbraceoffsetv)$)
        -- node[above=2pt] {#4} 
        ($(#1.north west)!(#1-1-#3.north east)!(#1.north east)+(0,\mymatrixbraceoffsetv)$);
}
\newcommand*\mymatrixbracebottom[4][m]{
    \draw[mymatrixbrace] ($(#1.south west)!(#1-1-#3.south east)!(#1.south east)-(0,\mymatrixbraceoffsetv)$)
        -- node[below=2pt] {#4} 
        ($(#1.south west)!(#1-1-#2.south west)!(#1.south east)-(0,\mymatrixbraceoffsetv)$);
}

%END MATRIX GUMF
 
 
 

\usepackage[mediumspace,mediumqspace,Grey,squaren]{SIunits}
\usepackage{subfigure}
 \usepackage{graphicx}
 \usepackage{mathtools}
 \usepackage{amsmath}
 \usepackage{setspace}
 \usepackage{tikz}

\newenvironment{enoomerate}{
\begin{enumerate}
  \setlength{\itemsep}{1pt}
  \setlength{\parskip}{0pt}
  \setlength{\parsep}{0pt}}{\end{enumerate}
}

\newenvironment{itemise}{
\begin{itemize}
  \setlength{\itemsep}{1pt}
  \setlength{\parskip}{0pt}
  \setlength{\parsep}{0pt}}{\end{itemize}
}



\newcommand{\up}[1]{\ensuremath{^{\textrm{#1}}}}
\newcommand{\down}[1]{\ensuremath{_{\textrm{#1}}}}

\newcommand{\Xangstrom}{\r{a}ngstr\"{o}m}


\newcommand{\Xth}{\up{\tiny{th}}}
\newcommand{\st}{\up{\small{st}}}
\newcommand{\nd}{\up{\small{nd}}}
\newcommand{\rd}{\up{\small{rd}}}

\newcommand{\Xeta}{\textit{et al }}
\newcommand{\Xiv}{\textit{in vivo }}
\newcommand{\Xxv}{\textit{ex vivo }}
\newcommand{\Xis}{\textit{in silico }}

\newcommand{\etal}{\textit{et al }}
\newcommand{\invivo}{\textit{in vivo }}
\newcommand{\exvivo}{\textit{ex vivo }}
\newcommand{\insilico}{\textit{in silico }}
\newcommand{\invitro}{\textit{in vitro }}

\newcommand{\um}{\micro m}

\newcommand{\xa}{\textit{a) }}
\newcommand{\xb}{\textit{b) }}
\newcommand{\xc}{\textit{c) }}
\newcommand{\xd}{\textit{d) }}
\newcommand{\xe}{\textit{e) }}

%DOCUMENT-SPECIFIC COMMANDS

 \newcommand{\pcatwo}{PCA\up{2} }




%\begin{figure}[htp]
%\centering
%\includegraphics[scale=0.5]{img/bubbles.png}
%\caption{Bubbles in the right atrium (RA) and vena cava (VC) of a guinea pig after decompression from 0.4MPa. From \cite{daniels1980detection}. Approximate scale: the area shown   in the image measures several mm across.}
%\label{bubbles}
%\end{figure}

%width = /textwidth

\begin{document}
\maketitle

\begin{center}
Supervisors: Alan Johnston and Peter McOwan
\\
$n$ words as counted by  the TeXcount script at \texttt{http://app.uio.no/ifi/texcount/index.html}.
\vspace{3cm}

%\includegraphics[scale=0.5]{img/twofaces.png}

\vspace{3cm}

\textit{Acknowledgements}\\
...
\end{center}


\pagebreak


\tableofcontents



\pagebreak

\section{Literature review}

\section{Terms}

\section{The overall goal}


\section{Modularity}

The property of modularity is the possibility to divide a system into multiple components!

\section{Forms of neural coding}

\begin{itemise}
\item \textbf{Single neuron activation}. The firing of a single neuron can convey binary information.
\item \textbf{Single spike frequency} can code a real-valued quantity.
\item \textbf{Spike frequency across multiple neurons} can code relative information between two real-valued quantities.
\item \textbf{Connection patterns} between neurons (the existence of a connection, or its strength) can code complex information, but this information cannot be extracted without activating the neurons and monitoring the outputs.
\end{itemise}

\section{Static and dynamic faces are processed differently}

The first evidence of a difference in the perception of expression between static and dynamic faces was found in 1991\cite{humphreys1993expression}.


\section{Identity vs. expression}

There is a substantial body of evidence that identity (information which is invariant within individuals) and expression (information which is invariant across perceived emotional states) are processed differently. On the high level, identity judgement and expression judgement have been observed to be doubly dissociated in prosopagnosics\cite{archer1994movement}. However, this observation may not allow us to generalise deductions to the normally-functional population, as prosopagnosics may have developed alternative recognition strategies such as non-holistic feature recognition (as is used to recognise classes of objects for which we do not possess a specialised representation or processing system).

On a slightly lower level, judgement reaction times differ depending on whether expression or identity is being judged; when judging identity, familiar faces are matched faster, but familiarity confers no advantage when judging expression\cite{bruce1986understanding}. This could imply that the computation of identity is intrinsically more complex or that other neural actions such as memory retrieval of biographical data are triggered.

On the lowest level, it is possible to find individual neurons which are receptive to either identity or expression\cite{hasselmo1989role}. Multidimensional scaling methods on their spike train data allow stimuli to be classified in either identity or expression space solely by neural response.

However, the location in one test subject of a small number of individual neurons which correlate with a particular condition provides no information about the algorithmics of face processing; it simply demonstrates that the brain can judge identity and expression at some level (which is intuitively obvious) and that this information can be coded by neural activation as opposed to connection patterning or higher-level codes such as spike train phase.

\section{Correlates between the two decouplings}

It is tempting to connect the identity-expression dichotomy with the static-dynamic dichotomy, as dynamic faces have constant identity but changing expression. This would be erroneous, as static faces can vary in both expression and identity.

\section{Object perception}

\section{Visual perception as dimensionality reduction}

Visual perception creates percepts from visual input. Photons arrive on the retina and induce signals in the optic nerve, which then pass to the LGN, dorsal and ventral visual pathways, and eventually effect conscious perception (such as when we perceive a face) or motor control (such as when we press a button to indicate that we have seen a face).

The number of photons arriving per unit time is so high that they cannot all be losslessly recorded, as shown by the reduced information capacity of the optic nerve\cite{wolff1993computing} compared to the retina, so information is compressed before dispatch. Motion representations are a simple form of compression; rather than recording the positions of a dot at each time-step (1,2,3,...,99,100), we can simply record its initial position (1) and speed (1 unit per second). Averaging is another simple compressor, as is nonlinear activation of cone cells (which require several afferent photons to change their membrane potential).


The bandwidth of the optic nerve is also smaller than that of incoming light signals, and this is dealt with by retinal adaptation.

%Insert from adaptation book here

These forms of compression can all be seen as transfer functions from low- to higher-level representations. The ultimate low-level representation of visual input is to record every photon arriving on the retina, but as this is impractical, optic nerve representations are compressed.

The process continues as we move further away from the retina and into the early visual system. Colour perception is another compression strategy, allowing any combination of wavelengths to be described by three coordinates in colour spaces like RGB, HSV or LAB.

Compression is evident in Marr's 










%the pca spacetime

%the thing needs to include more expressions! possibly let's use the emotional.

%the idea that we may have to process things in terms of the neural movements- we are very good at copying expressions, could this mean something? the input and output systems are likely to be linked, no?

\begin{singlespace}
\begin{footnotesize}
\begin{twocolumn}
\bibliographystyle{unsrt}
\bibliography{references,bib_litreview}
\end{twocolumn}
\end{footnotesize}
\end{singlespace}
\newpage

\input{appendices}

\end{document}



