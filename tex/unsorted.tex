\chapter{Unsorted}

\section{Characterising fire}

\section {Evolutionary value}

From an evolutionary point of view, mastery of fire was key to human development. Being able to control fire allowed early humans to cook food, defend themselves from predators and survive in cold, challenging environments. Fire was the first of a long line of technologies which release stored energy from fuel and turn it to human purposes; the earliest archaeological evidence of fire use dates back 1.8 million years, with frequent use found from 100,000 years ago\cite{bowman2009fire}. Even before this, hominids regularly encountered flame in the form of bushfires, although these were perceived as a threat, not a controllable, exploitable entity.

The evolving human visual system has therefore been exposed to a large amount of flamelike stimuli in the last 1.8 million years. These stimuli have often appeared in dangerous or life-threatening contexts, either posing a threat or aiding survival. In sufficiently extreme situations, such as extreme cold or heavy predation, those early humans who could successfully control fire had an increased chance of survival.

It is therefore natural to enquire whether the human visual system has become adapted in any way to the perception of flamelike stimuli. Does the visual system employ any specific representations or specialised models when attending to fire, does it use the same general-purpose systems employed when observing a novel moving stimulus?

This question recalls the ongoing debate concerning the specialisation of face perception. We find increased activation of the fusiform face area and inferior temporal sulcus while viewing faces\cite{allison2000social}; this can be explained either by innate specialisation or learned proficiency. In the same way, observation of fire may recruit neurons and systems which respond preferentially to, and perform better on, flame stimuli. On the other hand, observing fire may stimulate the same neural populations as observing other moving stimuli.

\section{High-level and low-level representations}

The goal of neuroscience is to impose structure and explanatory power on the neural systems present in the brain. This task is accomplished using descriptions in several different domains of representation:



\begin{figure}[htp]



    \renewcommand{\arraystretch}{1.8}

\begin{tabular}{ >{\bfseries}r | p{8cm}   }
& \textbf{Experiment 1}\\
\hline
  
	Design & 2AFC delayed match-to-sample (sample clip followed by two test clips)\\                   
  Stimuli & 1000-frame corpus \\
  Factors & sample length (10, 25, 50) frames or (0.2, 0.5, 1)seconds. \newline 
sample/test ratio (1.2 1.4 1.6 1.8 2).\\
  Block design & Sample length varied across blocks\newline
			Test length varied within blocks \newline
			15 conditions \newline
40 trials per condition \newline
600 trials \newline
25 training trials \\



\end{tabular}


\end{figure}

\begin{table}[htdp]
\caption{default}
\begin{center}
\begin{tabular}{|c|c|}

\end{tabular}
\end{center}
\label{default}
\end{table}%
