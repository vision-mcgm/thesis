\section{Literature review}

\section{Terms}

\section{The overall goal}


\section{Modularity}

The property of modularity is the possibility to divide a system into multiple components.

\section{Forms of neural coding}

\begin{itemise}
\item \textbf{Single neuron activation}. The firing of a single neuron can convey binary information.
\item \textbf{Single spike frequency} can code a real-valued quantity.
\item \textbf{Spike frequency across multiple neurons} can code relative information between two real-valued quantities.
\item \textbf{Connection patterns} between neurons (the existence of a connection, or its strength) can code complex information, but this information cannot be extracted without activating the neurons and monitoring the outputs.
\end{itemise}

\section{Static and dynamic faces are processed differently}

The first evidence of a difference in the perception of expression between static and dynamic faces was found in 1991\cite{humphreys1993expression}.


\section{Identity vs. expression}

There is a substantial body of evidence that identity (information which is invariant within individuals) and expression (information which is invariant across perceived emotional states) are processed differently. On the high level, identity judgement and expression judgement have been observed to be doubly dissociated in prosopagnosics\cite{archer1994movement}. However, this observation may not allow us to generalise deductions to the normally-functional population, as prosopagnosics may have developed alternative recognition strategies such as non-holistic feature recognition (as is used to recognise classes of objects for which we do not possess a specialised representation or processing system).

On a slightly lower level, judgement reaction times differ depending on whether expression or identity is being judged; when judging identity, familiar faces are matched faster, but familiarity confers no advantage when judging expression\cite{bruce1986understanding}. This could imply that the computation of identity is intrinsically more complex or that other neural actions such as memory retrieval of biographical data are triggered.

On the lowest level, it is possible to find individual neurons which are receptive to either identity or expression\cite{hasselmo1989role}. Multidimensional scaling methods on their spike train data allow stimuli to be classified in either identity or expression space solely by neural response.

However, the location in one test subject of a small number of individual neurons which correlate with a particular condition provides no information about the algorithmics of face processing; it simply demonstrates that the brain can judge identity and expression at some level (which is intuitively obvious) and that this information can be coded by neural activation as opposed to connection patterning or higher-level codes such as spike train phase.

\section{Correlates between the two decouplings}

It is tempting to connect the identity-expression dichotomy with the static-dynamic dichotomy, as dynamic faces have constant identity but changing expression. This would be erroneous, as static faces can vary in both expression and identity.